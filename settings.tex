% ****************************************************************************************************
% Settings script with new commands and certain packages

\usepackage{ifthen}
\newboolean{enable-backrefs} % enable backrefs in the bibliography
\setboolean{enable-backrefs}{false} % true false

% ****************************************************************************************************
% ****************************************************************************************************
% Setup, finetuning, and useful commands
% ****************************************************************************************************
\newcounter{dummy} % necessary for correct hyperlinks (to index, bib, etc.)
\newlength{\abcd} % for ab..z string length calculation
\providecommand{\mLyX}{L\kern-.1667em\lower.25em\hbox{Y}\kern-.125emX\@}
\newcommand{\ie}{i.\,e.\ }
\newcommand{\Ie}{I.\,e.\ }
\newcommand{\eg}{e.\,g.\ }
\newcommand{\Eg}{E.\,g.\ } 
\newcommand{\textunderscript}[1]{$_{\text{#1}}$}

% ****************************************************************************************************
% Loading some handy packages
% ****************************************************************************************************
\PassOptionsToPackage{utf8}{inputenc}	% latin9 (ISO-8859-9) = latin1+"Euro sign"
\usepackage{inputenc}				
\usepackage[british]{babel}

\PassOptionsToPackage{fleqn}{amsmath}		% math environments and more by the AMS 
 \usepackage{amsmath}
 \usepackage[mathletters]{ucs}

% Print UTF8 key
%http://tex.stackexchange.com/questions/83440/inputenc-error-unicode-char-u8-not-set-up-for-use-with-latex
\usepackage{stringenc}
\usepackage{pdfescape}

\PassOptionsToPackage{T1}{fontenc} % T2A for cyrillics
\usepackage{fontenc}     
\usepackage{textcomp} % fix warning with missing font shapes
\usepackage{microtype} % Added microtype to all
\usepackage{scrhack} % fix warnings when using KOMA with listings package
\usepackage{xspace} % to get the spacing after macros right  
\usepackage{mparhack} % get marginpar right
\PassOptionsToPackage{printonlyused,smaller,withpage}{acronym}
\usepackage[printonlyused, withpage]{acronym} % nice macros for handling all acronyms in the thesis

\usepackage{color, calc, blindtext}
\usepackage[compact]{titlesec} % compact white space under titles
\usepackage[pdftex]{graphicx} % Graphics options
%\usepackage{multicol} % multicolumn text
\usepackage{wrapfig} % figures wrapped in text
\usepackage{lineno} % Line Numbers 
\usepackage{epigraph} % For epigraphs in front of chapters
\usepackage[acronym]{glossaries}
% Provides a linked \doi{#1} that links doi:#1 to http://dx.doi.org/#1
% To change the text before the DOI, adjust this command   %\renewcommand\doitext{doi:}
\usepackage{doi}  

% Provides a linked \url{#1} that doesn't require escape characters
\usepackage{url}
\renewcommand{\UrlFont}{\normalsize}
% For \email{ADDRESS}, links ADDRESS to the url mailto:ADDRESS
% (uncomment to typeset the e\-/mail address in typewriter font;
%  otherwise, will be typeset in the \urlstyle above)
%\DeclareUrlCommand\emaillink{\urlstyle{tt}}
\providecommand*\emaillink[1]{\nolinkurl{#1}}
\providecommand*\email[1]{\href{mailto:#1}{\emaillink{#1}}}

% Counter for toc and caption 
\setcounter{tocdepth}{3} % Depth of sections to include in the table of contents - currently up to subsections 
\setsecnumdepth{subsection}% % Depth of sections to number in the text itself - currently up to subsections
\captionnamefont{\footnotesize} % Size of caption text
\captiontitlefont{\footnotesize} % Size of caption title

\usepackage{chngcntr} % To allow sublabels in appendix

% ********************************************************************                
% Colors
% ********************************************************************
\PassOptionsToPackage{dvipsnames,table,xcdraw}{xcolor}
	\RequirePackage{xcolor} % [dvipsnames] 
\usepackage{xcolor}
\definecolor{halfgray}{gray}{0.55} % chapter numbers will be semi transparent .5 .55 .6 .0
\definecolor{webgreen}{rgb}{0,.5,0}
\definecolor{webbrown}{rgb}{.6,0,0}
\definecolor{Maroon}{cmyk}{0, 0.87, 0.68, 0.32}
\definecolor{RoyalBlue}{cmyk}{1, 0.50, 0, 0}
\definecolor{Black}{cmyk}{0, 0, 0, 0}

% ********************************************************************                
% Colorboxes
% ********************************************************************
% Use package tcolorbox to provide little info boxes
% https://tex.stackexchange.com/questions/250069/create-a-color-box
\usepackage[many]{tcolorbox}
\usetikzlibrary{calc}
\definecolor{myblue}{RGB}{0,163,243}

\tcbset{mystyle/.style={
  breakable,
  enhanced,
  outer arc=0pt,
  arc=0pt,
  colframe=myblue,
  colback=myblue!20,
  attach boxed title to top left,
  boxed title style={
    colback=myblue,
    outer arc=0pt,
    arc=0pt,
    top=3pt,
    bottom=3pt,
    },
  fonttitle=\sffamily
  }
}

\newtcolorbox[auto counter,number within=chapter]{definitions}[1][]{
  mystyle,
  title=Box~\thetcbcounter,
  overlay unbroken and first={
      \path
        let
        \p1=(title.north east),
        \p2=(frame.north east)
        in
        node[anchor=west,font=\sffamily,color=myblue,text width=\x2-\x1] 
        at (title.east) {#1};
  }
}

% ****************************************************************************************************
% Setup floats: tables, (sub)figures, and captions
% ****************************************************************************************************
\usepackage{tabularx} % better tables
\setlength{\extrarowheight}{3pt} % increase table row height
\newcommand{\tableheadline}[1]{\multicolumn{1}{c}{\spacedlowsmallcaps{#1}}}
\newcommand{\myfloatalign}{\centering} % to be used with each float for alignment
\usepackage{caption}
\captionsetup{labelfont=bf,font = small} %hang makes it hang
\usepackage{subfig} 
\newsubfloat{figure} % Allow subfloats in figure environment
 
\usepackage{rotating} % Also allow sidewaystables
\usepackage{placeins} % For subfloating barriers
\usepackage{lscape} % for tables in landscape mode

\usepackage[hang]{footmisc} % additional footnote options
\usepackage{tablefootnote} % footnote for tables 
\usepackage{booktabs} % lines for tables
\usepackage{multirow}

% ****************************************************************************************************
% Setup code listings
% ****************************************************************************************************
\usepackage{listings} 
%\lstset{emph={trueIndex,root},emphstyle=\color{BlueViolet}}%\underbar} % for special keywords
\lstset{language=[LaTeX]Tex,%python,
%    keywordstyle=\color{RoyalBlue},%\bfseries,
    keywordstyle=\color{Black}, 
    basicstyle=\small\ttfamily,
    %identifierstyle=\color{NavyBlue},
%    commentstyle=\color{Green}\ttfamily,
    commentstyle=\color{Black}\ttfamily,
    stringstyle=\rmfamily,
    numbers=none,%left,%
    numberstyle=\scriptsize,%\tiny
    stepnumber=5,
    numbersep=8pt,
    showstringspaces=false,
    breaklines=true,
    frameround=ftff,
    frame=single,
    belowcaptionskip=.75\baselineskip
    %frame=L
} 
% ****************************************************************************************************
% PDFLaTeX, hyperreferences and citation backreferences
% ****************************************************************************************************
\PassOptionsToPackage{pdftex,hyperfootnotes=false,pdfpagelabels}{hyperref}
\usepackage{hyperref}  % backref linktocpage pagebackref

\pdfcompresslevel=9
\pdfadjustspacing=1 
\PassOptionsToPackage{pdftex}{graphicx}
\usepackage{graphicx} % for inserting figures
% ESO-PIC for figure placement on titlepage
\usepackage{eso-pic}
\newcommand\AtPageUpperRight[1]{\AtPageUpperLeft{%
   \makebox[\paperwidth][r]{#1}}}
% Support transparent figures
\usepackage{transparent}

% Setup the bibliography
%\PassOptionsToPackage{square,numbers}{natbib} % for natbib
\usepackage{natbib}				

\bibliographystyle{latex/ecology}
\setlength{\bibsep}{0pt plus 0.3ex} % Spacing between references

% ****************************************************************************************************
% Setup the style of the backrefs from the bibliography
% (translate the options to any language you use)
% ****************************************************************************************************
\newcommand{\backrefnotcitedstring}{\relax}%(Not cited.)
\newcommand{\backrefcitedsinglestring}[1]{(Cited on page~#1.)}
\newcommand{\backrefcitedmultistring}[1]{(Cited on pages~#1.)}
\ifthenelse{\boolean{enable-backrefs}}%
{%
		\PassOptionsToPackage{hyperpageref}{backref}
		\usepackage{backref} % to be loaded after hyperref package 
		   \renewcommand{\backreftwosep}{ and~} % separate 2 pages
		   \renewcommand{\backreflastsep}{, and~} % separate last of longer list
		   \renewcommand*{\backref}[1]{}  % disable standard
		   \renewcommand*{\backrefalt}[4]{% detailed backref
		      \ifcase #1 %
		         \backrefnotcitedstring%
		      \or%
		         \backrefcitedsinglestring{#2}%
		      \else%
		         \backrefcitedmultistring{#2}%
		      \fi}%
}{\relax}    

% Hyperreferences
\hypersetup{%
    %draft,	% = no hyperlinking at all (useful in b/w printouts)
    colorlinks=true, linktocpage=true, pdfstartpage=3, pdfstartview=FitV,%
    % uncomment the following line if you want to have black links (e.g., for printing)
    % colorlinks=false, linktocpage=false, pdfborder={0 0 0}, pdfstartpage=3, pdfstartview=FitV,% 
    breaklinks=true, pdfpagemode=UseNone, pageanchor=true, pdfpagemode=UseOutlines,%
    plainpages=false, bookmarksnumbered, bookmarksopen=true, bookmarksopenlevel=1,%
    hypertexnames=true, pdfhighlight=/O,%nesting=true,%frenchlinks,%
    %urlcolor=webbrown, linkcolor=RoyalBlue, citecolor=webgreen, %pagecolor=RoyalBlue,%
    urlcolor=black, linkcolor=black, citecolor=black, %pagecolor=Black,%
    pdftitle={\myTitle},%
    pdfauthor={\textcopyright\ \myName, \myUni, \myFaculty},%
    pdfsubject={},%
    pdfkeywords={\myKeywords},%
    pdfcreator={pdfLaTeX},%
    pdfproducer={LaTeX}%
}   

% ****************************************************************************************************
% MADSEN CLASS SETTINGS AND CHAPTER HEADER/STYLE
% ****************************************************************************************************

\usepackage{lmodern}
% https://tex.stackexchange.com/questions/436913/move-chapter-and-chapter-number-leftwards
\makechapterstyle{gilgauge}{%
\chapterstyle{default}
  \renewcommand*{\chapnamefont}{%
    \normalfont\Large\scshape\raggedleft}
  \renewcommand*{\chaptitlefont}{%
    \normalfont\Huge\bfseries\sffamily\raggedleft}
  \renewcommand*{\chapternamenum}{}
  \renewcommand*{\printchapternum}{%
    \makebox[36pt][l]{\hspace{0.4em}% <--- altered makebox value
      \resizebox{!}{4ex}{%
        \chapnamefont\bfseries\sffamily\thechapter}%
    }%
  }%
  \renewcommand*{\printchapternonum}{%
    \chapnamefont \phantom{\printchaptername \chapternamenum%
      \makebox[0pt][l]{\hspace{0.4em}%
        \resizebox{!}{4ex}{%
          \chapnamefont\bfseries\sffamily 1}%
      }%
    }%
    \afterchapternum %
  }% 
  \renewcommand*{\afterchapternum}{%
    \par\hspace{1.5cm}\hrule\vskip\midchapskip}}

% ****************************************************************************************************
% MEMOIR CLASS SETTINGS AND CHAPTER HEADER/STYLE
% ****************************************************************************************************
%\definecolor{chaptercolor}{gray}{0.35}
% helper macros
%\newcommand\numlifter[1]{\raisebox{-3cm}[0pt][0pt]{\smash{#1}}}
%\newcommand\numindent{\kern5pt}
%\newlength\chaptertitleboxheight
%\makechapterstyle{hansen}{
%  \renewcommand\printchaptername{\raggedleft}
%  \renewcommand\printchapternum{%
%    \begingroup%
%      \leavevmode%
%      \chapnumfont%
%      \strut%
%      \numlifter{\thechapter\hrulefill}%
%      \numindent%
%    \endgroup%
%  }
%  \renewcommand*{\printchapternonum}{%
%    \vphantom{
%    \begingroup%
%      \leavevmode%
%      \chapnumfont%
%      \numlifter{\vphantom{9}}%
%      \numindent%
%    \endgroup}
%    \afterchapternum
%  }
%  \setlength\midchapskip{0pt}
%  \setlength\beforechapskip{\baselineskip}
%  \setlength{\afterchapskip}{4\baselineskip}
%  \renewcommand\chapnumfont{%
%    \fontsize{5cm}{0cm}%
%    \bfseries% or 
%    \itshape
%    %\sffamily%
%    \color{chaptercolor}%
%  }
%  \renewcommand\chaptitlefont{%
%    \normalfont%
%    \Large%
%    \bfseries%
%    \scshape
%    \raggedright%
%  }%
%  \settototalheight\chaptertitleboxheight{%
%    \parbox{\textwidth}{\chaptitlefont \strut bg\\bg\strut}
%  }
%  \renewcommand\printchaptertitle[1]{
%    \raggedright\parbox[t][\chaptertitleboxheight][t]{.7\textwidth}{%
%      \chaptitlefont\strut ##1  \strut
%    }%
%  }
%}

% ****************************************************************************************************
% Setup autoreferences
% ****************************************************************************************************
% There are some issues regarding autorefnames
% http://www.ureader.de/msg/136221647.aspx
% http://www.tex.ac.uk/cgi-bin/texfaq2html?label=latexwords
% you have to redefine the makros for the 
% language you use, e.g., american, ngerman
% (as chosen when loading babel/AtBeginDocument)
% ****************************************************************************************************
\makeatletter
\@ifpackageloaded{babel}%
    {%
       \addto\extrasbritish{%
					\renewcommand*{\figureautorefname}{Figure}%
					\renewcommand*{\tableautorefname}{Table}%
					\renewcommand*{\partautorefname}{Part}%
					\renewcommand*{\chapterautorefname}{Chapter}%
					\renewcommand*{\sectionautorefname}{Section}%
					\renewcommand*{\subsectionautorefname}{Section}%
					\renewcommand*{\subsubsectionautorefname}{Section}% 	
				}%
       \addto\extrasngerman{% 
					\renewcommand*{\paragraphautorefname}{Absatz}%
					\renewcommand*{\subparagraphautorefname}{Unterabsatz}%
					\renewcommand*{\footnoteautorefname}{Fu\"snote}%
					\renewcommand*{\FancyVerbLineautorefname}{Zeile}%
					\renewcommand*{\theoremautorefname}{Theorem}%
					\renewcommand*{\appendixautorefname}{Anhang}%
					\renewcommand*{\equationautorefname}{Gleichung}%        
					\renewcommand*{\itemautorefname}{Punkt}%
				}%	
			% Fix to getting autorefs for subfigures right (thanks to Belinda Vogt for changing the definition)
			\providecommand{\subfigureautorefname}{\figureautorefname}%  			
    }{\relax}
\makeatother

% ****************************************************************************************************
% Final adjustments
% ****************************************************************************************************
\listfiles
%\PassOptionsToPackage{l2tabu,orthodox,abort}{nag}
%	\usepackage{nag}
%\PassOptionsToPackage{warning, all}{onlyamsmath}
%	\usepackage{onlyamsmath}

% classic thesis package for final layout
%\PassOptionsToPackage{eulerchapternumbers,%
%				 pdfspacing,floatperchapter,linedheaders,%
%				 subfig,eulermath,parts,
%				 dottedtoc}{classicthesis}						 

% Available options for classicthesis.sty 
% drafting
% parts nochapters linedheaders
% eulerchapternumbers beramono eulermath pdfspacing minionprospacing
% tocaligned dottedtoc manychapters
% listings floatperchapter subfig
%\usepackage{latex/classicthesis} 

% ****************************************************************************************************
% Changing the text area
% ****************************************************************************************************
%\linespread{1.05} % a bit more for Palatino
%\areaset[current]{312pt}{761pt} % 686 (factor 2.2) + 33 head + 42 head \the\footskip
%\setlength{\marginparwidth}{7em}%
%\setlength{\marginparsep}{2em}%

% Change the margins to the University of Sussex default format
\usepackage[a4paper,top=2.5cm,bottom=2.5cm,left=4cm,right=2cm,headsep=10pt]{geometry}

% For correct spacing throughout the document
\newcommand{\linespacing}{1.5}
\renewcommand{\baselinestretch}{\linespacing}
\setlength{\footnotemargin}{3mm}

% Set page style and headers
\renewcommand{\sectionmark}[1]{\markboth{}{\small{\thesection\ #1} } } % with number of section a

\makepagestyle{custom}% Create custom page style
\makeoddhead{custom}% Adjust odd header for custom page style
{\leftmark}% Left odd header
{\thepage}% Center odd header
{\rightmark}% Right odd header
\makeheadrule{custom}{\textwidth}{.5pt}% Header rule width/thickness for custom page style
\aliaspagestyle{chapter}{custom} % just to save some space
\copypagestyle{chapter}{plain}
\makeoddhead{chapter}{}{\thepage}{}
\makeoddfoot{chapter}{}{}{}

% ****************************************************************************************************
% Using different fonts
% ****************************************************************************************************
%\usepackage[oldstylenums]{kpfonts} % oldstyle notextcomp
\usepackage[osf]{libertine}
%\usepackage{fourier} % font
%\usepackage{hfoldsty} % Computer Modern with osf
%\usepackage[light,condensed,math]{iwona}
%\renewcommand{\sfdefault}{iwona}
%\usepackage{lmodern} % <-- no osf support :-(
%\usepackage[urw-garamond]{mathdesign} <-- no osf support :-(


% ****************************************************************************************************
% Mute some warnings
% ****************************************************************************************************
\setlength{\headheight}{14pt} % to avoid multiple "\headheight is too small" warnings 
\setlength{\parskip}{5pt} % to have some space between paragraphs
\pdfminorversion=5  % to avoid warnings like "PDF inclusion: found PDF version <1.6>, but at most version <1.5> allowed"  

\graphicspath{{./figures/}} % directory with all the pictures
\flushbottom
