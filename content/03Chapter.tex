\chapter{ Impacts of past abrupt land change on local biodiversity globally}
%\addcontentsline{toc}{chapter}{Chapter 3}
\markboth{}{Impacts of past abrupt land change on local biodiversity globally}

Abrupt land change, such as deforestation or agricultural intensification, is a key driver of biodiversity change. Following abrupt land change, local biodiversity often continues to be influenced through biotic lag effects. However, understanding of how terrestrial biodiversity is impacted by past abrupt land changes is incomplete. By combining geographically- and taxonomically-broad data on local biodiversity with quantitative estimates of abrupt land change detected within time series of satellite imagery from 1982 to 2015, here we show that abrupt land change in the past continues to influence present species assemblages globally. Species richness and abundance were reduced by 4.2\% and 2\%, respectively, and assemblage composition was altered at sites with an abrupt land change compared to unchanged sites, although impacts differed among taxonomic groups. Biodiversity recovered to levels comparable to unchanged sites after >10 years. Ignoring delayed impacts of abrupt land changes likely results in incomplete assessments of biodiversity change.

\section{Introduction}
Natural and anthropogenic processes change the terrestrial surface of the Earth \citep{Ellis2013,Song2018}, which have been shown to impact biodiversity \citep{Newbold2015,Jung2018} and ecosystem services \citep{Isbell2015}. Previous studies have found that present differences in land-surface conditions reduce local biodiversity globally \citep{Gibson2011,Newbold2015}. However, these studies often ignore the impacts of past land change \citep{Foster2003,Watson2014}. Simulations and experiments have demonstrated that land changes of greater magnitude have larger impacts on the number of species and individuals \citep{Dornelas2010,Hautier2015,Santini2016}. Yet, few studies have quantified the impacts of land change in the past on local biodiversity globally.

Local biodiversity continues to be influenced by past land change through biotic lags. Biotic lags—including ecological processes such as extinction debt \citep{Tilman1994,Kuussaari2009,Halley2016}, colonization credit \citep{Hylander2013} and ecological memory effects \citep{Ogle2015}—negatively affect the number of species and individuals present within local assemblages \citep{Halley2016,Jung2018,Perring2018}, and potentially reduce resilience \citep{Hautier2015,Nimmo2015}. The impacts of land change on species assemblages through biotic lag depend on species’ abilities to persist \citep{Turner1998} and recover \citep{Martin2013,Fu2017,Moreno-Mateos2017}. Most previous global studies \citep{Supp2014,Fu2017,Moreno-Mateos2017,Shackelford2017} investigating abrupt land changes in the past have used descriptive study-specific categories of “land changes”, e.g. wild fire, flooding or cultivation, thus hindering comparisons among studies, and preventing predictions. To assess the impacts of abrupt land change on local biodiversity more generally, comparable quantitative measures of “land change” are needed.

The availability of time series of satellite imagery enables the detection and quantification of land changes globally \citep{Kennedy2014,Song2018}. Land change can be defined as abrupt shifts in intra- and inter-annual dynamics of remotely-sensed photosynthetic activity quantified through vegetation indices \citep{Linderman2005,Pettorelli2005}. Abrupt shifts in magnitude \citep{Kennedy2012,Watson2014,DeVries2015b} and/or trend \citep{DeJong2013} of photosynthetic activity, and the time passed since such shifts \citep{POTTER2003,Kennedy2012} are three key attributes of land change \citep{Watson2014}. Several algorithms have been developed to detect abrupt land change \citep{Zhu2017} and measure these attributes. However, attributes of remotely-sensed abrupt land change have never before been used to assess biotic lags in local biodiversity.

Here we investigate the impacts of abrupt land change in the past—defined as the single largest shift in magnitude and/or trend of photosynthetic activity \citep{Verbesselt2010a,DeJong2013,Song2018}—on local biodiversity globally. We used data on local biodiversity of unprecedented geographic and taxonomic coverage from the Projecting Response of Ecological Diversity in Changing Terrestrial Systems (PREDICTS) database \citep{Hudson2016}. At each site, where local biodiversity was sampled, we assessed time series of high spatial resolution (nominal \textasciitilde30m) Landsat satellite imagery from 1982-2015 for the presence of an abrupt land change (Figure \ref{F03_01}\textbf{a}) and, where detected, we quantified key attributes, \ie, shifts in magnitude, trend and time passed. Using hierarchical analyses, we compared four measures of local biodiversity (species richness, total abundance, evenness and species turnover) between paired sites (5,563 sites with and 10,102 without an abrupt land change) from 377 studies (Figure \ref{F03_01}\textbf{b}). We expect that abrupt land changes with larger shifts in magnitude and trend have greater impacts on local biodiversity through biotic lag effects and that with more time passed local biodiversity can recover from the impacts of abrupt land change.

% ---------------- Figure 1 --------------------- %
\begin{figure}[h]
\centering
\includegraphics[width=1\textwidth]{chapter3/F01}
\caption{ \textbf{Examples and distribution of sites without and with abrupt land change}. (\textbf{a}) Remotely-sensed time series of monthly Enhanced Vegetation Index (EVI; green points) at an unchanged site; a site with an abrupt shift in magnitude, \ie, loss in EVI; and a site with a shift in EVI trend, \ie, an increase in annual EVI. Linear (black lines) and seasonal (dark green lines) fits of the change detection algorithm \citep{Verbesselt2010a} are shown. (\textbf{b}) Location of 5,563 sites from 377 studies in the PREDICTS database with an abrupt land change in the monitoring period (since 1982) of the Landsat 4-8 missions. Colours indicate sites with abrupt land change that had a magnitude gain (\textbf{+}) or loss (\textbf{-}) and/or trend increase (\textbf{+}) or decrease (\textbf{-}); (magnitude | trend). For ease of viewing, the location of 10,102 sites without an abrupt land change has been omitted. Latitudinal distribution of sites with an abrupt land change and time passed between abrupt land change and biodiversity sampling (in years, mean and standard deviation shown in red) by 25\textdegree  latitudinal bands. Map shown in Eckert IV equal-area projection.}
\label{F03_01}
\end{figure}
% -------------------------------------------- %

\section{Methods}
\subsection{Biodiversity data} 
We used data from the Projecting Responses of Ecological Diversity In Changing Terrestrial Systems (PREDICTS) database(Hudson et al. 2017), which includes species’ presence and abundance data from ‘studies’ with at least two spatially-explicit ‘sites’, information on the date of sampling, and local land-use and/or land-use intensity(Hudson et al. 2017). We simplified the original PREDICTS land use and land-use intensity information(Hudson et al. 2014, 2017) by allocating each site to one of three broad land-use categories: primary vegetation (PV, i.e. primary [non-] forest), secondary vegetation (SV, i.e. mature, intermediate, young and indeterminate age secondary vegetation) or human-dominated vegetation (HDV, i.e., plantation forest, cropland, pasture, urban). Studies were grouped into eight broad taxonomic groups based on the sampled taxa: plants, fungi, ground dwelling invertebrates (e.g., soil-fauna, snails, beetles), flying invertebrates (e.g., butterflies, bees, dragonflies), amphibians, reptiles, birds or mammals. 

We assessed four measures of local biodiversity that complement each other and have previously been shown to be sensitive to abrupt land change(Supp & Ernest 2014; Santini et al. 2017). For each site in the PREDICTS database, we calculated within-sample species richness and, where data on abundance were available, log10 total abundance of individuals, adjusted by sampling effort following Newbold et al.(Newbold et al. 2014). After visual inspection, we removed one outlier study (a study of soil biomass, ID $“DL1_2012__CalvinoCancela”$) from further analyses because of very large abundance estimates (> 3×10⁶ individuals). As a measure of assemblage evenness, we calculated the arcsine square root transformed probability of an interspecific encounter (PIE), which quantifies the probability of two individuals randomly chosen from an assemblage representing different species(Hurlbert 1971). As a measure of turnover in species assemblage composition within studies, we calculated the Sørensen similarity index among spatial pairs of sites within each study and land-use category(Magurran 2004).

Species assemblages were sampled at various spatial extents defined by each study’s sampling method and land use. Following the PREDICTS data curation protocol we assumed the allocated land use to be dominant within the reported sampling extent (maximum linear extent [MLE], in meters) of each site(Hudson et al. 2014, 2017). For studies without reported MLE (4779 sites, 18.3\% of all sites), we used either the mean MLE for each taxonomic group and corresponding sampling method, e.g., mist netting, pitfall trapping, or the mean MLE within the same taxonomic group. To test whether these interpolated MLEs are consistent among taxonomic groups and sampling method, we randomly removed 25\% of the reported MLEs and found the interpolated MLEs to be reasonably correlated (Pearson’s r = 0.73, p < 0.001). We included all studies with a MLE < 3000m (98.3\% of all sites), approximately 100 times the nominal resolution (~30m) of the remotely-sensed data used in this study, and removed four studies with sites located in water (rivers, coastal areas or ponds), identified by intersecting all sites with a global permanent water surface mask(Pekel et al. 2016), as a precaution as sites within these studies likely have low positional accuracy. To spatially link species assemblage with remote sensing data, we calculated a rectangular buffer with MLE as radius (MLEmean= 412.1 m ± 1661.82 m SD) around each site’s coordinates as the smallest area that fully captures all grid cells of varying sampling units (e.g., point counts, line transects). 

\subsection{Remote sensing data} 
We used land-surface reflectance products derived from the sensors of the Landsat 4 (1982 – 1993), 5 (1984 - 2012), 7 (1999 – ongoing), and 8 (2013 – ongoing) missions available within Google Earth Engine (GEE)(Gorelick et al. 2017), based on raw United States Geological Service Landsat Collection images (Tier 1) to calculate the Enhanced Vegetation Index (EVI, as two-band version(Jiang et al. 2008)) as a proxy of photosynthetic activity. We masked all cloud-covered grid cells (~30 m nominal resolution) using the cloud-detection output in the ‘cfMask’ band(Zhu & Woodcock 2012) and removed occasional snow- and water-covered grid cells, i.e. those with negative EVI values. All data preparation and extraction were performed within GEE(Gorelick et al. 2017).

For each Landsat image and PREDICTS site we calculated the mean EVI within the rectangular buffer (y ̅) and extracted time series of all EVI values. We removed outliers introduced by satellite sensor errors, missed cloud shadows or bad quality estimates by calculating the absolute difference of all y ̅ values from the median absolute deviation (MAD) per EVI time series(Leys et al. 2013). EVI values more than a conservative threshold of two units of deviation away from the MAD or in the top 1\% of all MAD estimates were set to NA(Leys et al. 2013). Time series of EVI data were temporally aggregated to monthly maximum value composites to ensure equal intervals between data points and to reduce the amount of noise and missing data. Because of the ongoing consolidation of the global Landsat archive(Wulder et al. 2016), there can be periods of consecutively missing data, particularly before the launch of Landsat 7 in 1999 (Supplementary Fig. 4a). To remove gaps of ≥ 5 years of consecutively missing data, which might affect the precision of land change attribute calculations, we identified and then truncated time series to include only the years from 1999 onwards in subsequent analyses (see Supplementary Fig. 4b). In total 25,656 sites had suitable EVI time series, with an average 18.83 ($\pm$ 6.7 SD) years duration containing on average 1.82 years ($\pm$ 1.57 SD) of consecutively missing data.

\subsection{Abrupt land change detection} 
To identify the presence of abrupt land change and its attributes in EVI time series, we used the Breaks For Additive Season and Trend (BFAST) algorithm34 modified to work with missing data and optimized to find the single most influential abrupt land change in a time series(de Jong et al. 2013). BFAST accurately detects abrupt land changes(Verbesselt et al. 2010a; DeVries et al. 2015) by using a multiple regression model to estimate both trend and seasonal components of a time series(de Jong et al. 2013):  y ̅_t=α_s+β_s t+ ∑_(p=1)^k▒γ_p   sin⁡(2πpt/h+ δ_p )+ε_t, where y ̅_t is the mean EVI at time t, s the segment in the time series, α the intercept, β the slope (i.e., trend), p and k the order of the seasonal term (k = 2), γ the amplitude, δ the phase and ε the residual error. The expected frequency to detect an abrupt land change in a time series is determined by h and, following previous studies(Verbesselt et al. 2010a, b), was set as the ratio of the number of data points per year (12 months) to the total length of the individual time series (in months). Whenever the inclusion of the seasonal component caused the model to fail to converge (17\% of all fitted models), we removed the seasonal component by time series decomposition (‘stlplus’ package(Hafen 2016)) prior to fitting BFAST with a trend component only. BFAST detects abrupt land change when model residuals depart significantly (p < 0.05) from a statistical boundary(Zeileis 2005). To test for significant departure we used two complementary approaches(Zeileis 2005; Verbesselt et al. 2010b, a) using first, a moving sum of residuals (MOSUM) test within the monitoring period (determined by h) and second, an information-theoretic approach, the Bayesian Information criterion (BIC). All BFAST models were fitted using the ‘bfast’ package (ver. 1.5.7) in R (ver. 3.5)(Verbesselt et al. 2010a; R Core Team 2018).

For the single most influential abrupt land change detected in each time series, we calculated the relative shift in magnitude as the immediate change in EVI [((y ̂_j  -y ̂_(j-1) ))/(〖|y ̂〗_(j-1) |), where y ̂_j is the first monthly estimate of y ̅ predicted by the BFAST model after an abrupt land change has been identified and y ̂_(j-1) the predicted estimate one month before], the difference in linear trend as increase/decrease in EVI before and after the abrupt land change (β_after-β_before, where βafter and βbefore are the predicted linear trends in EVI from the BFAST model, before and after the abrupt land change), and the time passed (in months, t_n - t_j) between the date of the abrupt land change (t_j) and the start of biodiversity sampling (t_n). Attributes of abrupt land change were grouped into bins as follows (Supplementary Fig. 1 and Table 1): for shifts in magnitude (> 50\%, > 25\% and <= 50\%, and <= 25\% EVI loss or gain, Supplementary Fig. 1a), for shifts in trend (0.01, 0.05, and > 0.05 lower or higher EVI trend change, Supplementary Fig. 1b) and time passed (<5, 5-10, and >10 years ago, Supplementary Fig. 1c). The three attributes of abrupt land change were only marginally correlated among each other (mean Pearson’s |r| < 0.07, Supplementary Fig. 5). Sites without an abrupt land change detected by BFAST are referred to as “unchanged” sites (0) and all studies containing only unchanged sites (10,196 sites of 262 studies) were excluded from further analyses.

\subsection{Statistical analyses} 
We built hierarchical models comparing biodiversity measures between paired sites without and with an abrupt land change in the past. Hierarchical generalized linear mixed effects (LME) models were fitted separately for species richness (using a Poisson error distribution), total abundance, and the PIE (using a Gaussian error distribution). For models of species richness we included an observation-level random effect (i.e., site ID) to account for overdispersion(Harrison 2015). For each LME model we compared several candidate random-effect structures by fitting null models with combinations of different random intercepts and random slopes to determine the structure with the lowest overall Aikake Information Criterion (AIC). Random effects always included the study ID to account for study-level differences in sampling methods, optionally a spatial block ID in which sites were located, the site’s land-use category (PV, SV, HDV), the presence of an abrupt land change (yes|no), as well as the studies climatic zone (tropical, arid, temperate or continental climate) according to the Koeppen Geiger classification(Peel et al. 2007). Whenever a climatic zone could not be determined (for instance on small islands), we attributed studies to a zone based on latitude and a site’s terrestrial biome (1369 sites). The most parsimonious random-effect structure by AIC was identical among response variables and included – besides the study ID – the spatial block and land-use category as random intercept as well as the presence of an abrupt land change as random slope. We included the binned attributes of abrupt land change, e.g. shifts in magnitude, trend, and time passed, as fixed effects in our models with the unchanged sites (0) as paired reference comparison. Separate models were fitted for each taxonomic group using the direction (positive or negative) of magnitude and trend shifts because of limited data availability. Full LME models were tested for significant differences (p < 0.05) from a null model using likelihood ratio tests, while significant differences between bins were approximated by Wald statistics(Bates et al. 2015). To compare impacts of a shift in magnitude against shift in trend, we assessed the difference in Akaike’s Information criterion (AIC), a difference of ∆AIC <7 commonly indicates little improvement in model fit, and calculated ordinary Pearson correlation coefficients between their effects as models were otherwise not comparable because of equal fixed structures. All models were fitted using the ‘lme4’ package (ver. 1.1-14 in R ver. 3.5)(Bates et al. 2015; R Core Team 2018). 

To estimate differences in species assemblage composition we calculated the mean compositional similarity (as quantified by the Sørensen similarity index) between all pairs of sites without and with an abrupt land change in the same study and land-use category. To visualize the mean similarity for each land change attribute bin, we performed hierarchical complete-linkage clustering (‘hclust’ function in R) on Manhattan distances between estimates of compositional similarity transformed relative to the mean difference between pairs of unchanged sites.

\section{Results}

\clearpage
%\bibliography{content/04Chapter}

%\appendix
%\begingroup
%  % SI - Figure 1 Missing data
\begin{figure}[h]
\centering
\includegraphics[width=1\textwidth]{chapter3/SI01}
\caption{ Average temporal distribution of Landsat data and an example times series of Landsat data. (\textbf{a}) Distribution of available Enhanced Vegetation Index (EVI) data in years covered by the Landsat missions. Points show the average monthly EVI data availability per year (0 to 12 months of data) across time series and PREDICTS sites grouped by 15\textdegree latitude bins. The size of points indicates the mean data availability (0 to 100\% with 100\% having 12 months of available data in a given year), while the colour shows the number of PREDICTS sites contributing to the mean (as PREDICTS sites were sampled in varying years). (\textbf{b}) Example time series for one PREDICTS site with a high proportion of missing data before 1999. In all analyses such time series were truncated to the period from 1999 onwards (indicated by the dashed line).}
\label{SI03_01}
\end{figure}

% SI - Figure 2 Binning
\begin{figure}[h]
\centering
\includegraphics[width=1\textwidth]{chapter3/SI02}
\caption{ Number of sites with abrupt land change per attribute. Number of sites (black line) per attribute of abrupt land change with (\textbf{a}) the relative shift in magnitude, (\textbf{b}) the shift in trend as difference in annual EVI trend, and (\textbf{c}) the time passed between abrupt land change and biodiversity sampling. Background colours in (\textbf{a}) and (\textbf{b}) indicate the binning into six groups for shifts in magnitude (> 50\%, > 25\% to $\leq$ 50\%, and $\leq$ 25\% EVI loss [$---$ to $-$] or gain [$+++$ to $+$]), and in trend (0.01, 0.05, and > 0.05 annual negative [$---$ to $-$] to positive [$+++$ to $+$] EVI trend differences). Gray lines in (\textbf{c}) delineate bins of time passed ($\leq$ 5 years, > 5 and $\leq$ 10 years, and >10 years). Colours as in Figure \ref{F03_02}.}
\label{SI03_02}
\end{figure}

% SI Table 1------ %
% From here https://www.tablesgenerator.com/
\begin{table}[]
\centering
\caption{Number of PREDICTS sites and studies with an abrupt land change. Shown as either a change in magnitude (columns) and/or change in trend (trend). Symbols as in Figure \ref{F03_02}. }
\label{SIT03_01}
\begin{tabular}{@{}lllllllllll@{}}
                                          &                                           & \multicolumn{7}{c}{\textbf{Shift in magnitude}}                                                                                                                                                                    &                               &                             \\
                                          &                                           & \textbf{- - -}             & \textbf{- -}                & \textbf{-}                   & \textbf{0}                    & \textbf{+}                   & \textbf{+ +}                & \textbf{+ + +}              & \textbf{Total sites}          & \textbf{Studies}            \\ \cmidrule(l){3-11} 
                                          & \multicolumn{1}{l|}{- - -}                & 2                          & 8                           & 192                          & NA                            & 73                           & 26                          & 22                          & \cellcolor[HTML]{EFEFEF}323   & \cellcolor[HTML]{C0C0C0}57  \\
                                          & \multicolumn{1}{l|}{- -}                  & 7                          & 281                         & 642                          & NA                            & 497                          & 158                         & 53                          & \cellcolor[HTML]{EFEFEF}1638  & \cellcolor[HTML]{C0C0C0}175 \\
                                          & \multicolumn{1}{l|}{-}                    & 7                          & 88                          & 256                          & NA                            & 231                          & 154                         & 53                          & \cellcolor[HTML]{EFEFEF}789   & \cellcolor[HTML]{C0C0C0}184 \\
                                          & \multicolumn{1}{l|}{0}                    & NA                         & NA                          & NA                           & 10102                         & NA                           & NA                          & NA                          & \cellcolor[HTML]{EFEFEF}10102 & \cellcolor[HTML]{C0C0C0}358 \\
                                          & \multicolumn{1}{l|}{+}                    & 9                          & 102                         & 399                          & NA                            & 410                          & 205                         & 49                          & \cellcolor[HTML]{EFEFEF}1174  & \cellcolor[HTML]{C0C0C0}237 \\
                                          & \multicolumn{1}{l|}{\textbf{+ +}}         & 47                         & 172                         & 342                          & NA                            & 465                          & 254                         & 86                          & \cellcolor[HTML]{EFEFEF}1366  & \cellcolor[HTML]{C0C0C0}224 \\
\multirow{-7}{*}{\textbf{\rotatebox{90}{Shift in trend}}} & \multicolumn{1}{l|}{\textbf{+ + +}}       & 12                         & 137                         & 47                           & NA                            & 34                           & 12                          & 31                          & \cellcolor[HTML]{EFEFEF}273   & \cellcolor[HTML]{C0C0C0}56  \\
                                          & \multicolumn{1}{l|}{\textbf{Total sites}} & \cellcolor[HTML]{EFEFEF}84 & \cellcolor[HTML]{EFEFEF}788 & \cellcolor[HTML]{EFEFEF}1878 & \cellcolor[HTML]{EFEFEF}10102 & \cellcolor[HTML]{EFEFEF}1710 & \cellcolor[HTML]{EFEFEF}809 & \cellcolor[HTML]{EFEFEF}294 &                               &                             \\
                                          & \multicolumn{1}{c|}{\textbf{Studies}}     & \cellcolor[HTML]{C0C0C0}34 & \cellcolor[HTML]{C0C0C0}135 & \cellcolor[HTML]{C0C0C0}246  & \cellcolor[HTML]{C0C0C0}358   & \cellcolor[HTML]{C0C0C0}263  & \cellcolor[HTML]{C0C0C0}171 & \cellcolor[HTML]{C0C0C0}83  &                               &                            
\end{tabular}
\end{table}
% ------ %

% SI - Figure 3 Cross-correlations
\begin{figure}[h]
\centering
\includegraphics[width=1\textwidth]{chapter3/SI03}
\caption{ Correlations between attributes of abrupt land change. Showing shifts in magnitude, trend and time passed (see Methods). The lower facets show a point density plot, the upper facets the Pearson correlation coefficient between pairs of attributes and the diagonal a density plot.}
\label{SI03_03}
\end{figure}

% SI - Figure 4 violin plots
\begin{figure}[h]
\centering
\includegraphics[width=1\textwidth]{chapter3/SI04}
\caption{ Distribution of time passed between abrupt land changes and start of biodiversity sampling. Shown for (\textbf{a}) shifts in magnitude, and (\textbf{b}) shift in trend bins. Colours as in Figure \ref{F03_02}. Black dots and error bars show the mean $\pm$ one standard deviation. Number of sampled sites per bin are shown above each bin.}
\label{SI03_04}
\end{figure}



%\endgroup
