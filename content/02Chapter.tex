\chapter{Chapter 2}
%\addcontentsline{toc}{chapter}{Introduction}
\markboth{}{Local species assemblages are influenced more by past than current dissimilarities in photosynthetic activity}

\section{Abstract}
Most land on Earth has been changed by humans and past changes of land can have lasting influences on current species assemblages. Yet few globally representative studies explicitly consider such influences even though auxiliary data, such as from remote sensing, are readily available. Time series of satellite-derived data have been commonly used to quantify differences in land-surface attributes such as vegetation cover, which will among other things be influenced by anthropogenic land conversions and modifications. Here we quantify differences in current and past (up to five years before sampling) vegetation cover, ,and assess whether such differences differentially influence taxonomic and functional groups of species assemblages between spatial pairs of sites. Specifically, we correlated between-site dissimilarity in photosynthetic activity of vegetation (the Enhanced Vegetation Index) with the corresponding dissimilarity in local species assemblage composition from a global database using a common metric for both, the Bray-Curtis index. We found that dissimilarity in species assemblage composition was on average more influenced by dissimilarity in past than current photosynthetic activity, and that the influence of past dissimilarity increased when longer time periods were considered. Responses to past dissimilarity in photosynthetic activity also differed among taxonomic groups (plants, invertebrates, amphibians, reptiles, birds and mammals), with reptiles being among the most influenced by more dissimilar past photosynthetic activity. Furthermore, we found that assemblages dominated by smaller and more vegetation-dependent species tended to be more influenced by dissimilarity in past photosynthetic activity than prey-dependent species. Overall, our results have implications for studies that investigate species responses to current environmental changes and highlight the importance of past changes continuing to influence local species assemblage composition. We demonstrate how local species assemblages and satellite-derived data can be linked and provide suggestions for future studies on how to assess the influence of past environmental changes on biodiversity.

\section{Introduction}
Throughout the Earth’s history, land has changed constantly by a combination of natural and anthropogenic forces. Palaeontological evidence indicates that humans have transformed approximately 75 % of the land at least once (Ellis et al. 2010, Ellis 2011), with changes in many land-surface attributes, such as vegetation cover,  accelerating since the beginning of the industrial revolution (Lambin and Geist 2006, Steffen et al. 2015). Changes in vegetation cover may be caused by climatic factors, such as CO2 fertilization or altered precipitation patterns (Zhu et al. 2016), or anthropogenically caused land conversions, such as deforestation, re- and afforestation (Dupont et al. 2003, Hansen et al. 2013, Müller et al. 2014) or land modifications, such as degradation, intensification (Gibbs and Salmon 2015, Rufin et al. 2015) or return to less intensive forms of land use (Zomer et al. 2016). Over time, these changes have shaped both land and species assemblages in complex ways (Foster et al. 2003, Watson et al. 2014, Perring et al. 2016).
Most global meta-analyses investigating the influence of differences in vegetation cover on species assemblages have assumed that any difference in vegetation cover at the time of biodiversity sampling is the dominant influence (Stein et al. 2014, Newbold et al. 2014, 2015, Alroy 2017). However, this assumption might be incorrect as assemblages can be heavily influenced by legacy effects of past changes in vegetation cover (Foster et al. 2003, Watson et al. 2014, Ogle et al. 2015, Perring et al. 2016). For the recent past (e.g., up to five years prior to biodiversity sampling), ecological memory or carry-over effects, i.e. the capacity of past events to influence present and future ecological assemblages (Harrison et al. 2011, O’Connor et al. 2014, Ogle et al. 2015), have been proposed as mechanisms that shape species assemblages. These effects can arise through site-specific environmental factors, for instance altered conditions because of agricultural practices (Perring et al. 2016, 2018) or different sequences and successional recovery from changes in past vegetation cover (Johnson and Miyanishi 2008, Walker et al. 2010, Watson et al. 2014). No detailed global analysis to date has explicitly considered the influence of both current and past differences in vegetation cover on current species assemblages.
While some differences in species assemblages can be traced back to changes in vegetation cover in the late quaternary (Vegas-Vilarrúbia et al. 2011, McMichael et al. 2017), there is some evidence that changes in vegetation cover in the more recent past can influence plant (Jakovac et al. 2016), invertebrate (Valtonen et al. 2013) or vertebrate assemblages (Newton et al. 2014, Cole et al. 2015, Graham et al. 2017). However, this has – —to our knowledge—  not been assessed comparatively across multiple taxonomic groups. Furthermore, it is likely that species with specific traits, such as certain body size and/or trophic level, may be differentially affected by past changes in vegetation cover because of differences in their metabolic rate (for animals), longevity or dispersal abilities (Sutherland et al. 2000, Brown et al. 2004, Speakman 2005, Thomson et al. 2011, De Palma et al. 2015). Depending on the type and magnitude of a past changes in vegetation cover (as a proxy for land-surface changes) plant assemblages can either be dominated by small, fast sprouting  or taller, nutrient-demanding species (Jakovac et al. 2016, Perring et al. 2018). Until now, our understanding of the influence of past differences in vegetation cover on species assemblages has been limited to case studies focused on specific regions or certain taxonomic and functional groups. However, a recently published globally representative dataset on species assemblages of broad taxonomic coverage (Hudson et al. 2017) and globally available satellite-derived data enable us to consider explicitly both current and past differences in land-surface attributes.
	Satellite-derived data can provide internally consistent estimates of how land differs across time and space (Pettorelli et al. 2005, Kennedy et al. 2014). LLand-surface attributes such as photosynthetic activity of vegetation can be quantified using spectral indicators from satellite-derived data (Gamon et al. 1995, Zhang et al. 2006). Changes in photosynthetic activity of vegetation can be related to both climatic (Fensholt et al. 2012, Zhu et al. 2016) and anthropogenic factors such as land conversions and modifications (Lambin et al. 2003, Mueller et al. 2014). Subtle differences in vegetation dynamics (as measured by various satellite-derived vegetation indices), such as faster greening rate or differing seasonal amplitude, between years have been used to characterize land change (Lambin and Strahler 1994, Linderman et al. 2005, Lupo et al. 2007). Recent studies have used such differences to identify changes in land use such as in pasture use intensity (Rufin et al. 2015), fallow periods in croplands (Estel et al. 2015, Tong et al. 2017), small-scale deforestation (DeVries et al. 2015) and broad scale land degradation and intensification (de Jong et al. 2011, Mueller et al. 2014).  Dissimilarity metrics describing the entirety of recent land history (e.g. including both differences in land use and land cover as well as climatic and site-specific factors) can be calculated between spatial pairs of time series as the overall dissimilarity in photosynthetic activity (Linderman et al. 2005, Lhermitte et al. 2011). Increasingly such methods have been linked to dissimilarity in local species assemblage composition (Rowhani et al. 2008, Goetz et al. 2014, Nieto et al. 2015, Hobi et al. 2017), however few studies have explicitly distinguished between current and past dissimilarity in photosynthetic activity.
	Here we use a time series dissimilarity metric (the Bray-Curtis index) to quantify dissimilaritys in a land-surface attribute, e.g. photosynthetic activity of vegetation, among spatial pairs of sites in the Projecting Responses of Ecological Diversity In Changing Terrestrial Systems (PREDICTS) dataset (Hudson et al. 2017). We explicitly distinguish between dissimilarity in current and past photosynthetic activity (BCEVI­) among spatial pairs of, defined here as the five years prior to the ‘current’ year, and assess how they influence compositional dissimilarity (BCBiodiversity­) between species assemblages among paired sites. This pairwise comparison approach allows us to investigate (i) the overall influence of past relative to current dissimilarity in photosynthetic activity on species assemblages where we hypothesize that the influence of past dissimilarity increases with longer past periods considered. Furthermore, we investigate (ii) whether different taxonomic groups respond differently to past dissimilarity in photosynthetic activity, and (iii) if species with particular functional characteristics, i.e., those that are smaller and/or more vegetation-dependent, are more affected by past dissimilarity in photosynthetic activity than others. 

\section{Data and Methods}
\subsection{Remotely-sensed data}
A temporal profile of spectral reflectance values was derived from the Moderate Resolution Imaging Spectroradiometer (MODIS) sensor on board NASA’s Terra and Aqua satellites. Since the year 2000, MODIS has provided continuous spectral data of medium-scale resolution (nominal ~500 m resolution) with high temporal revisit rates (a global image collection is taken every day) (Schaaf et al. 2002). We used the Bidirectional Reflectance Distribution Function and Albedo (BRDF) product (MCD43A4.005), which aggregates the highest quality daily spectral reflectance values into 8-day composites of seven spectral bands (Schaaf et al. 2002). Google Earth EngineTM was used to download and process temporal profiles of all spectral bands for each site (Gorelick et al. 2017). We calculated a spectral index measuring photosynthetic activity (the two-band Enhanced Vegetation Index – EVI; Jiang et al. 2008), which is based on a ratio between the near-infrared (nir, 841-876 nm) and red (red, 620-670 nm) spectral band (). We used the EVI as it has been designed to reduce atmospheric contamination and not to saturate in high biomass regions such as tropical rainforests (Huete et al. 2002, Jiang et al. 2008). We applied the following pre-processing steps (also see flowchart in Supplementary Fig. 1) to the nir and red BRDF bands individually to fill missing observations and filter out extreme data points. 
First, we detected and removed extreme outliers in the BRDF data that may have been introduced by cloud shadows, atmospheric haze, inversion errors or sensor failures. We calculated the absolute difference of all values from the median relative to the total median absolute deviation (MAD) of all values (Leys et al. 2013). Pixels which were more than a conservative threshold of two units deviation (but see Leys et al. 2013) away from the MAD as well as greater than 99 % of all other difference values were set to missing. This data-defined threshold removed only the most extreme outliers and retained fluctuations that are within the bounds of median conditions. We chose this procedure rather than using the MODIS BRDF quality data set (stored in the separate MCD43A2.005 product) to maintain the maximum number of observations assuming that bad quality inversions of the BRDF product are filtered and smoothed out by subsequent pre-processing steps.
Second, we interpolated missing values using a Kalman filter, a smoother for estimating missing data points based on preceding data (Kalman 1960). Previous studies have shown that Kalman filters perform well in filling gaps in BRDF time-series especially in data-poor regions (Samain et al. 2008). The best model for the Kalman filter for a given time-series was estimated using the “forecast” R package (“auto-arima” function) by selecting the model with the lowest Akaike Information Criterion (AIC) (Hyndman and Khandakar 2008). We only interpolated consecutive gaps ≤ 40 days (i.e. five consecutive 8-day BRDF composites) as longer interpolations would reduce our ability to detect short-term changes in photosynthetic activity.  We excluded all time-series from further analyses with more than 50 % remaining missing data (average proportion of missing data = 6.32 ± 10.31 %) in the time period considered (see Supplementary Fig. 2). 
Lastly, we applied a Savitzky-Golay filter (filter length = 5, “signal” R-package; Signal developers, 2014) to reduce the amount of random noise remaining in the time series, but retain small abrupt changes that might occur (Joensson and Eklundh 2004). The Savitzky-Golay filter performs well relative to other smoothing techniques in removing noise (Kandasamy et al. 2013). Our pre-processing steps aimed to remove influential outliers and random noise from each time series, but we cannot rule out that some non-informative noise has remained in the time series. From these pre-processed BRDF data we calculated the EVI for each 8-day composite (Jiang et al. 2008).

\subsection{Species assemblage data}
We used data on species’ abundance within local-scale assemblages from the PREDICTS database (Hudson et al. 2017; downloaded on 3 February 2016, see Supplementary Fig. 1), which is the largest global database investigating anthropogenic impacts on terrestrial species assemblages to date. The PREDICTS database has collated local-scale species assemblage records from the published literature (henceforth “sources”) comparing observations among at least two localities (henceforth “sites”) with differing land use or related pressures. Sources in the PREDICTS database having multiple sampling methodologies and taxonomic groups were split accordingly into different “studies”. Wherever sampling effort differed among sites within a study, we followed the approach of Newbold et al. (2014) and adjusted abundance values assuming that recorded abundance increase linearly with effort. Each study was assigned to one of six higher taxonomic groups based on the sampled species identity (Plants, Invertebrates, Amphibians, Reptiles, Birds and Mammals). We grouped plants and invertebrates into single individual groups as there were insufficient data to divide them into smaller groups (e.g., functional divisions such as flying vs ground-living insects). Studies of fungi were dropped from the analyses because of insufficient data.
	Of the 25224 sites with abundance data, we removed 6109 sites because their sampling durations spanned more than a year or because the start of biodiversity sampling differed by more than three months among sites within a study. This was done to avoid seasonal effects confounding any link between species assemblage composition and remote-sensing derived estimates. Furthermore, we removed 10248 sites from studies that sampled biodiversity before the 18th of February 2006 to ensure MODIS data availability for at least five years prior to biodiversity sampling. We chose to use a five-year period to allow sufficient MODIS coverage (since year 2000) for the majority of studies in the PREDICTS database (median biodiversity sampling start date = 2007-07-17). In total 8867 sites were suitable to be linked with MODIS remote-sensing data.
	The spatial extent of biodiversity sampling at PREDICTS sites typically differs from the resolution of MODIS data. We used the Maximum Linear Extent (MLE) information within the PREDICTS database, which summarises the spatial extent of sampling within a study in metres (Hudson et al. 2017). Sites from a few studies had large MLE (up to 40 km) and after visual exploration, we decided to keep only those sites that were within the 99 % quantile of all MLE values (MLE < Q99 = 3000 m, removing 249 sites). Some studies had missing MLE information (25 % of all studies with abundance data, 728 sites), where no MLE estimate could be obtained during after the PREDICTS data curation (Hudson et al. 2017). We filled  missing MLE information with the average MLE estimate of each taxonomic group with corresponding sampling method, and any remaining missing MLE, for which no other combination of taxonomic group and sampling method had MLE estimates, with the average MLE for each taxonomic group. We tested the robustness of this assumption by removing 25 % of the existing MLE estimates at random and found interpolated MLE values to be reasonably accurate (r = 0.73, p < 0.001). We used the centre coordinates for the rest of the sites (mean MLE ± SD = 256.52 m ± 437.93 m) as their spatial extent roughly matched the nominal spatial resolution of the MODIS data (~500 m). 
We excluded studies from our analyses where all study sites fell within a single MODIS grid cell, to suit our hierarchical statistical approach (see below). Some sites within a study could fall into the same MODIS grid cell, therefore for all further analyses we randomly selected one site per study per grid cell 100 times (See section on analysis – pairwise differences below for description of permutation procedure and Supplementary Fig. 3 for a schematic), resulting in 100 different subsets that we used for all further analyses. Our final dataset included data from 198 studies with 4053 sites per permutation and model covering all major continents and most taxonomic groups (Fig. 1a).

\subsection{Species trait compilation}
A species’ size and trophic level are two of the most basic traits for understanding differences in species assemblage structure (Speakman 2005, Terborgh 2015). We classified studies into size and trophic bins based on a simple majority: small (>0-9 g animal body mass or > 0-9 cm plant height), medium (10 – 99 g or 10-99 cm) or large species (>= 100 g or >= 100 cm), or predominantly herbivore, omnivore, carnivore or detritivore species, by estimating the dominant number of species (simple sum of measurement) within a study. Studies with species of predominantly unknown size or trophic level were removed from the analysis. We thus classified entire studies to the dominant bins as each study’s methodology would likely constrain the average size of animals or plants that can be observed. Data on average adult body mass (in g) were collected for mammals (Jones et al. 2009) and birds (Myhrvold et al. 2015), while for plants we used height (in cm) data from the TRY database (Kattge et al. 2011). The estimates of species trophic levels originate from Kissling et al. (2014), Wilman et al. (2014) and other sources of the literature for invertebrates (mostly collected by L. Bentley). For species for which size or trophic level data were unavailable, we used the genus-wide average for size and the most common trophic level (at least 95 % of all species with data within a genus). We excluded studies (N=8) from further analyses where no clear majority of species (> 50 %) could be assigned to one of the bins (Supplementary Fig. 4), leaving a total of 65 studies with size information and 130 studies with trophic information. 

\subsection{Analysis - Pairwise dissimilarity}
We linked dissimilarity in photosynthetic activity of vegetation with compositional dissimilarity in species assemblages globally. Specifically, we examined the differential influence of “current”  as the 365 days prior to species assemblage sampling) and “past” (, the i years prior to the current year, where i = 1,..,5) dissimilarity in photosynthetic activity between spatial pairs of sites (Fig 1b, Supplementary Fig 3). We separately considered past periods of increasing lengths (in years, so , , ). For example, if species assemblage sampling was conducted from the 1st of April until the 15th of July 2008,  was the 365 days prior to 1st of April 2008, i.e. 1st April 2007 – 31th March 2008, and past i years as the period (number of full years i) before April 1st 2007.
	We used the pairwise Bray-Curtis (BC) index, frequently used in community ecology studies, as a metric to quantify dissimilarity in species assemblage composition between sites (Bray and Curtis 1957, Faith et al. 1987, Su et al. 2004). We also considered the binary version of the BC index, the Sørensen similarity index, to assess whether our results are robust to metric choice. The BC index is a modified Manhattan distance, where the summed distances between values are standardised by the summed values of each site, thus quantifying pairwise dissimilarity from 0 (completely similar) to 1 (entirely different). We used the BC index to measure compositional dissimilarity in local species assemblages (BCBiodiversity) between sites within a PREDICTS study. We also applied the BC index to the EVI time series (BCEVI) to characterize the dissimilarity between sites in (inter- and intra-annual) vegetation dynamics measured through a proxy representing photosynthetic activity of vegetation in current ( and past years (), which to our knowledge is the first time the BC index has been applied to assess dissimilarity between remotely-sensed time series.
	The BC index is calculated between two pairs of sites with PREDICTS species assemblage records or two EVI time-series from sites x and y as follows: 
For species assemblages, x and y are the abundances of observed species (n = total number of species) at both sites (non-occurring species were assumed to be absent and set to zero), while for the EVI time series x and y are observed EVI values on the same date (n = total number of dates) in the time series at both sites. The BCEVI was calculated on either single or multiple years () of EVI time series (Figure 1b, Supplementary Fig. 3).
	Compared to other metrics quantifying dissimilarity between time-series based on remotely-sensed data (Lhermitte et al. 2011) the BCEVI index has the advantages of (a) taking the actual spectral values as well as distance between them into account, meaning it can be compared between different land-cover types, and (b) using the same method for assessing dissimilarity between species assemblages and between remote-sensing observations. In remote-sensing terms, for any vegetation index (such as EVI), the BCEVI index can be interpreted as a measure of absolute differences between two sites in the amount and timing of photosynthetic activity scaled by the total amount of photosynthetic activity available. By calculating the BCEVI index on temporal profiles of EVI measurements, we incorporate all differences in EVI between two sites into a single dissimilarity metric. No further scaling has been done as range and unit of the BCEVI index values were identical for current and past BCEVI.

\subsection{Analysis - Statistical modelling}
The aim of the statistical modelling is to estimate the influence of current and past BCEVI on the BCBiodiversity (Fig. 1b). For different time periods (0-5 years) we estimated this influence using separate models rather than an interaction term as current and past BCEVI were highly collinear (Random permutation pick: Pearson’s r > 0.9, df = 4046, p < 0.001). A hierarchical modelling approach using generalized linear mixed models (GLMMs) with Gaussian link function was used to fit models of current and past BCEVI independently for each time period, taxonomic group, size and trophic bins. GLMMs account for differing sampling methodologies among the PREDICTS studies, by including the “study” as a random intercept in all models. We also allowed the effect of current and past BCEVI to vary for each study by incorporating it as a random slope. From each model, we obtained the fixed effects (estimated slope) of the predicted BCBiodiversity per unit of current and past BCEVI.
	As we are primarily interested in the influence of past BCEVI (of different periods) on differences in BCBiodiversity, we incorporated the influence of current BCEVI by transforming the average past BCEVI effects (across all permutations) relative to current effects ( – 1). The resulting ratio describes whether the explicit influence of past BCEVI on BCBiodiversity is larger (> 0) or smaller (< 0) than the influence of current BCEVI (Figure 2). The precision estimates (predicted standard errors) of the effect of past BCEVI were also transformed relative to the precision estimates of current BCEVI (). This helps to visually assess the added imprecision after accounting for the imprecision already present in current BCEVI. 
	Estimating pairwise comparisons in any regression model would imply substantial pseudo-replication. To account for this, we took the subdiagonal of 100 permuted site-by-site matrices (Supplementary Fig. 3) to construct the GLMMs of 100 separate permutations. This ensures that for each fitted GLMM, our pairwise comparisons are mutually independent subsets (Longacre et al. 2005, Newbold et al. 2016). Fixed effects and standard errors for both current and past BCEVI were averaged across all permutations. Furthermore, for each model we calculated a marginal and conditional pseudo R-square (Nakagawa and Schielzeth 2013) and significance estimate (Halekoh and Højsgaard 2014), and averaged them across all permutations. As for the fixed effects and precision estimates, the differences in explained marginal variance of past BCEVI were assessed relative to the explained marginal variance of current BCEVI.
	All analyses were performed in R (ver 3.2.2, R Core Team 2015) using lme4 (ver. 1.10, (Bolker et al. 2009, Bates et al. 2015) for modelling, and vegan (ver. 2.2.3, Oksanen et al. 2015) for the BC calculation of species assemblages data. The processed MODIS data and R-code for the analyses are available on GitHub (https://github.com/Martin-Jung/PastLandSurfaceConditions). 

\section{Results}
The compositional dissimilarity of species assemblages (BCBiodiversity) increased with between-site dissimilarity in current and past photosynthetic activity (BCEVI; current: β = 0.289, βSE = 0.063, p < 0.001; past yr1:5:  β = 0.334, βSE = 0.07, p < 0.001; Fig. 2, Supplementary Fig. 5). When the influence of past BCEVI was assessed relative to current BCEVI, the BCBiodiversity between sites was more pronounced – although the imprecision also increased - when longer time periods (of up to five years) of past BCEVI were considered (Fig. 3). Furthermore, the consideration of past BCEVI calculated up to five years prior to current BCEVI increased the relative explained marginal variance by 16.7 % (Supplementary Table 1). We ensured that the BC index was robust with regards to varying time period lengths (Supplementary Fig. 6), spatial autocorrelation (Supplementary Fig. 7) and other temporal and geographic biases (Supplementary Fig. 8). Similar results were found by using a different metric of species assemblage composition, the Sørensen similarity index, that does not require species abundance estimates (Supplementary Fig. 9).
	The influence of past BCEVI on species assemblages was found to vary among taxonomic groups and time periods considered (Fig. 4). Dissimilarity in plant, invertebrate, reptilian and bird assemblage composition increased with increasing BCEVI of the past two to five years. In contrast, the influence of past BCEVI on mammalian assemblages was greatest for the first two years relative to the influence of current BCEVI but decreased when longer periods of three to five years of past BCEVI were considered. Meanwhile, amphibian assemblages were more influenced by current than past BCEVI between sites (Fig. 4).
	The influence of past BCEVI differed with respect to body size (Fig. 5). Species assemblages that were dominated by small- (> 0-9 g body mass) and medium-sized (10-99 g) mammals were more influenced by differences in BCEVI over the past one to three years, while the influence on assemblages dominated by larger (>= 100 g) mammals increased with longer time periods. Compared to assemblages dominated by medium-sized birds, assemblages of large bird species were up to five times more influenced by past relative to current BCEVI. For plant assemblages with available information on size, we found that assemblages dominated by medium-sized plants were more influenced by past BCEVI compared to those assemblages dominated by larger plant species (Fig. 5). 
	Differences among trophic levels were also seen in the influence of past BCEVI on BCBiodiversity and increased with longer time periods considered (Fig. 6). Species assemblages dominated by omnivorous and herbivorous assemblages were more influenced by past BCEVI of even one year relative to the influence of current BCEVI, while detritivores assemblages were only more influenced by past BCEVI if periods of the past three years were considered (Fig. 6). In contrast, studies with predominantly carnivorous species were more influenced by current BCEVI and showed no overall trend with longer time periods of past BCEVI considered (Fig. 6). 

\section{Discussion}
The main aim of this study was to investigate if between-site dissimilarity in current and past photosynthetic activity of vegetation (BCEVI ) can predict compositional dissimilarity in sites’ species assemblages (BCBiodiversity). In contrast to previous PREDICTS-based studies that used discrete measures of current land use and land-use intensity (Newbold et al. 2015, 2016), we used a continuous measure of between-site dissimilarity in remotely-sensed photosynthetic activity that summarises (inter- and intra-annual)  vegetation dynamics in a single metric (the BCEVI). We explicitly differentiated between current (the full year prior to species assemblage sampling) and past BCEVI (periods of up to five years before current) that could have influenced compositional dissimilarity in species assemblages. Similar to previous studies using the same dataset to analyse compositional differences with respect to land use (Newbold et al. 2016), we found that sites with more different current BCEVI also had more different species assemblages (Fig. 2, Supplementary Fig. 5). However, the BCEVI calculated over five years prior to biodiversity sampling had, on average, an even greater influence on between-site dissimilarity in species assemblage composition compared to current BCEVI (Fig. 3). This pattern was consistent across most taxonomic (Fig. 4) and functional groups (Fig. 5 and 6). Here we discuss potential causes and implications of the observed patterns as well as factors that can affect the BCEVI.

\subsection{Potential drivers of dissimilarities in photosynthetic activity }


\clearpage
%\bibliography{content/01Chapter}

%\appendix
%\begingroup
%  \input{content/01AChapter}
%\endgroup
