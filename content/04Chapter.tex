\chapter{Incorporating land-cover changes between 1992 and 2015 into biodiversity projections }
%\addcontentsline{toc}{chapter}{Chapter 4}
%\markboth{}{Incorporating land-cover changes between 1992 and 2015 into biodiversity projections }
\label{C04}

Changes in global land cover are important factors that determine past and present biodiversity patterns. It has been proposed that attributes of land-cover change, such as the time passed or the sequence in land cover \textendash\ \ie from forest to agriculture \textendash\ likely affect local biodiversity differently. Local biodiversity might continue to be affected by past land-cover change depending these attributes, which need to be considered to assess biodiversity change, especially in global and national biodiversity projections. Yet, the impacts of attributes of past land-cover change on local biodiversity have not been fully determined globally and most existing biodiversity projections remain largely uninformed of past land-cover change. Here, we combine time series of annual land cover from the period 1992 to 2015 with data of local biodiversity globally. Using hierarchical models comparing sites with and without a land-cover change in the past, we ask whether biodiversity differences vary with the time passed or the sequence after a land-cover change occurred and how this affects global and national biodiversity projections. Overall, we found local biodiversity to be consistently lower in sites with a past land-cover change. However, with increasing time passed after land-cover change local biodiversity recovered to levels comparable to unchanged sites. Furthermore, depending on the land-cover sequence, we observed either increases or decreases in local biodiversity and we demonstrated how a consideration of past land-cover change affects global and national biodiversity projections, especially so in tropical and economically developing countries. Our findings have implications for global biodiversity models given that most past and future projections of biodiversity ignore lasting influences of past land-cover changes. 

\section{Introduction}
\label{C04_01}

The terrestrial surface of the Earth is shaped by natural and anthropogenic processes \citep{Foley2005}. The outcomes of these processes alter soil, plant and human structures, which collectively define terrestrial land cover \citep{DiGregorio2000,Lambin2006}. Land cover \textendash\ quantified as either continuous or categorical estimate of the Earth land surface conditions \textendash\ is commonly derived from remotely-sensed spectral measurements with many studies having mapped the distribution of land-cover categories globally \citep{DeFries1994,Hansen2000,Tuanmu2014,Grekousis2015}. Knowledge of land-cover change is important to help understand biodiversity change and create future projections and scenarios \citep{Harfoot2014,Titeux2016,Newbold2016,Kehoe2017a}. Yet, with exception of vegetation \citep{Hansen2013,Song2018}
or water-covered areas \citep{Pekel2016}, few temporally consistent estimates of global land-cover change exist.

Quantifying change in remotely-sensed land cover is challenging. For continuous representations of land cover, remotely-sensed changes are commonly detected by exploiting differences in timing, amplitude and direction of remotely-sensed spectral measurements \citep{Coppin2004,Lhermitte2011,Zhu2017}. There have been initial attempts to incorporate land-cover changes detected from these differences into categorical land-cover maps \citep{Zhu2014,Hermosilla2018}, but the majority of land-cover maps remain uninformed of preceding land cover. Quantifying temporal change in categorical representations of land cover has been problematic because of inconsistencies in thematic resolution that lead to unrealistic estimates of land-cover change \citep{VERBURG2011,Cardille2016,Abercrombie2016}. A new generation of temporally consistent time series of land cover \citep{ESA2017,Hermosilla2018,Nowosad2018,Sulla-Menashe2019} are beginning to emerge that allow the investigation of land-cover change globally and its impacts on biodiversity.

Biodiversity is impacted by past and present differences in land cover \citep{Newbold2015,Newbold2016,Jung2018}. Local species richness has been estimated to be up to 31\% lower globally in the most anthropogenically modified land compared to “primary vegetation” sites \citep{Newbold2015}. However most previous global studies have considered only differences in land use and/or land cover at the time of biodiversity sampling \citep{Gibson2011,Murphy2014,Newbold2015}, thus ignoring lasting influences of past changes in land cover. There is evidence that the occurrence and abundance of species is not only determined by differences in present but also past land cover \citep[, Chapter \ref{C03}] through so called ‘biotic lag’ effects, such as ecological memory effects \citep{Ogle2015} or extinction debts \citep{Kuussaari2009}. The impact of past land-cover change on biodiversity likely depends on certain attributes such as magnitude and time passed since land-cover change \citep[Chapter \ref{C03}, ][]{Martin2013,Watson2014,Fu2017} or the sequences of land-cover \citep{Watson2014,Nowosad2018}. 

Land-cover change causes varying sequences of land cover \citep{Nowosad2018}, which often have differing impacts on local biodiversity \citep{Foster2003}. \cite{Bremer2010} reported an average loss of species richness globally for land changing from grass- or shrubland to forest cover, but not for land changing from secondary vegetation to forest cover. Meanwhile, biodiversity in secondary vegetation has been shown to recover more quickly if land was previously covered by grassland rather than agriculture \citep{Dyer2010}, although among taxonomic groups, especially plant diversity, abundance and growth have been shown to be influenced by lasting influences of an agricultural past \citep{Chazdon2003,Fraterrigo2006,DeFrenne2010,Perring2018}. Other studies have highlighted the lasting effect that changes in forest \citep{Gonzalez2016} or wetland cover \citep{Halstead2014} might have on biodiversity. While these studies suggest that land-cover sequences need to be considered for explaining differences in local biodiversity, little is known about the influence of land-cover sequences across taxonomic groups and at global and national scales, which could affect projections of biodiversity.

To guide decision making, projections of global and national biodiversity change are often useful to inform policy \citep{Pereira2010,Visconti2014}. Biodiversity projections can be used to create scenarios of biodiversity change in response to pressures such as land change \citep{Newbold2015,Newbold2016,Titeux2016}, which can inform science-policy platforms \citep{Harfoot2014,Visconti2014,Purvis2018} like as the Intergovernmental Platform on Biodiversity and Ecosystem Services (IPBES). However most existing biodiversity projections ignore lasting effects of past land-cover change. This is especially problematic for tropical, developing nations, where much land has been converted from forest to agriculture or pasture covered land in recent decades \citep{Curtis2018} and that are recognized as global biodiversity hotspots \citep{Brooks2002,Laurance2014}. Under a business-as-usual scenario of future biodiversity change, especially less economically developed countries will suffer the greatest losses in local biodiversity \citep{Newbold2015,Visconti2014}, however these projections might \textendash\ depending on attributes of land-cover change \textendash\ over- or underestimate impacts on biodiversity.

The overall aim of this study is to investigate (\textit{i}) how local biodiversity is impacted by a land-cover change in the past as derived from a global remotely-sensed land cover product, (\textit{ii}) if impacts on local biodiversity differ with attributes of land-cover change such as differing sequences of land cover or time passed \citep{Watson2014}, and (\textit{iii}) how particularly differing sequences of land cover affect global and national biodiversity projections. Overall, this study adds to our knowledge of how attributes of land-cover change affect local biodiversity and demonstrates how these attributes can be incorporated into global and national biodiversity projections.   
