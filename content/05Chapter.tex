\chapter{Landscape-wide land changes correlate with, but rarely explain local bird diversity change}
%\addcontentsline{toc}{chapter}{Chapter 3}
%\markboth{}{Impacts of past abrupt land change on local biodiversity globally}
\label{C05}

There is an ongoing debate whether biodiversity at local scales is declining and what might drive this change. Land changes are suspected to impact local biodiversity change. However, there is little evidence across spatial and temporal scales and for multiple functional groups of species, thus limiting our understanding of the causes of local biodiversity change. Here we investigate whether landscape-wide land changes, opposed to those at the local scale, are driving local bird diversity change. We link time series of 34 years of breeding bird survey (BBS) data (1984-2017) at 2745 routes across the continental United States of America with remotely-sensed satellite imagery (\textasciitilde30m resolution) from the Landsat missions. Specifically, we assessed for each year what proportion of the landscape surrounding the BBS routes had a land change \textendash\ defined as abrupt shift in magnitude or trend of photosynthetic activity as detected by the Breaks for Additive Seasonal and Trend (BFAST) algorithm \textendash\ and tested whether large proportions of concomitant or preceding landscape-wide land changes explain changes in bird diversity, quantified as either geometric mean of relative abundance (GM) or progressive Bray-Curtis index (pBC). We found that the GM was negatively and the pBC positively correlated with a large proportion of land changes in the wider landscape. Furthermore, the consideration of preceding \textendash\ instead of concomitant \textendash\ landscape-wide land changes explained on average more variation in bird diversity change. Overall, landscape-wide land changes failed to explain most of the variation in local bird diversity change for most BBS routes regardless if bird diversity change is differentiated by functional groups or geographic regions. This study is one of the first studies attempting to link land and biodiversity change. It highlights the influence of preceding and concomitant land change on biodiversity and makes suggestions for promising directions of future research.  

\section{Introduction}
\label{C05_01}

Ongoing human alteration of the Earth surface causes changes in biodiversity across scales \citep{Gibson2011,Murphy2014,Newbold2015}. Globally, about 32\% of all known vertebrate species show decreasing population sizes and range contractions \citep{Ceballos2017,WWF2018} with reported species extinction rates being several times higher than expected naturally \citep{Brooks2002,Pimm2014}. Yet, any change in biodiversity is scale and measure dependent \citep{Sax2003,Chase2013} and, perhaps surprisingly, there is still a debate whether local \textendash\ opposed to global \textendash\ biodiversity is truly changing \citep{Thomas2013,McGill2014}. 

A number of global meta-analyses demonstrated that some biodiversity measures, notably species richness, have not changed at the local scale \citep{Vellend2013,Vellend2017,Dornelas2014}. However, these results have been questioned, particularly on whether the data are spatially and temporally biased \citep{Gonzalez2016} or if sites with and without land change were differentiated \citep{Cardinale2018}. This raises the question whether changes on land can explain changes in local biodiversity measures across space and time. 

Present differences on land influence local biodiversity globally. Previous studies found local biodiversity to be consistently reduced at sites with more intensively used land \citep{Murphy2014,Newbold2015,Alroy2017}, where on average 13.6\% fewer species and 10.7\% fewer individuals were observed compared to undisturbed primary vegetation \citep{Newbold2015}. However, these analyses relied on spatial comparisons of local biodiversity and therefore do not capture temporal biodiversity change. In addition, they ignored the influence of past land changes \citep{Perring2018,Jung2018} and did not consider landscape-wide land changes, which can influence local biodiversity \citep{Tscharntke2012,Turner2015,Miguet2015}. 

Local biodiversity is influenced by the variability of resources, such as food or nesting material, or through ecological processes, such as migration or fear of predation, at the landscape scale \citep{Hanski2000,Chase2003,Turner2015,Fernandez2016}. However these influences are not static and landscapes are constantly changing because of natural and anthropogenic factors \citep{Pickett1985,Manning2009,Turner2015}. Previous studies have shown that landscape-wide land changes may have a lasting influence on local biodiversity through ‘biotic lag’ effects \citep{Metzger2009,Ewers2013}. Yet, most studies focussed on small geographic regions and changes in forest cover \citep{Rittenhouse2010} and did not investigate general impacts of landscape-wide land changes on local biodiversity across spatio-temporal scales. A lack of data on local biodiversity and landscape-wide land change has so far prevented comparative assessments \citep{DePalma2018}.

Increasing availability of satellite imagery enables the assessment of land change at broad spatial and temporal scales \citep{Kennedy2014,Pasquarella2016}. Long-running satellite missions, such as Landsat, provide one of the best sources to monitor land-surface conditions \citep{Kennedy2014,Vogelmann2016,Hermosilla2018,Song2018}. Time series of land-surface conditions, such as photosynthetic activity, can measure intra- and inter-annual vegetation dynamics \citep{Pettorelli2005,Fisher2006} and specific algorithms have been developed to detect land changes as changes in photosynthetic activity \citep{Verbesselt2010,Zhu2017}. Land changes can be differentiated by attributes \citep{Watson2014}, such as abrupt shifts in magnitude, causing an immediate loss or gain of vegetation \citep{DeVries2015b}, or shifts in trend, causing either greening or browning over time \citep{DeJong2013,Muller2014}. These attributes can be robustly quantified at the landscape scale and linked to changes in local biodiversity.

Birds are one of the best surveyed taxonomic groups globally. Local biodiversity change quantified from repeated breeding bird surveys (BBS) has been widely studied \citep{Harrison2014,Pardieck2018}. Previous studies have shown that changes in bird diversity are dependent on the specific biodiversity measure     considered \citep{Schipper2016,Jarzyna2017} and are often non-linear \citep{Gutzwiller2015,Barnagaud2017}. Bird diversity change also varied spatially \citep{Harrison2014,Jarzyna2017} with many birds of particular functional traits, such as migratory or grassland dependent species, declining in developed countries \citep{Fewster2000,Sanderson2006,Stanton2018}. Land changes are most likely a driving factor of these declines \citep{Harrison2014,Harrison2016}, and yet most studies investigated only spatial correlations between remotely-sensed attributes of land change and local bird diversity \citep{Rowhani2008,Goetz2014,Hobi2017}. Notably \cite{Rittenhouse2010} found bird assemblage composition to be altered in landscapes with more “disturbed forests”, which they assessed using remotely-sensed time series. However, to our knowledge, no previous study has investigated whether landscape-wide land changes correlate with and explain changes in local bird diversity.

Consequently, this study hypothesizes that (\textit{i}) changes in local bird diversity are driven by landscape-wide land changes varying by land change attributes, (\textit{ii}) local bird diversity change can best be explained by past land changes, and (\textit{iii}) the explanatory power of landscape-wide land changes on local bird diversity change varies across geographic regions and functional groups of bird species. We combine 34 years (1984–2017) of annual BBS records collected at sites across the continental United States of America with time series of medium-high resolution (nominal \textasciitilde30m) satellite imagery from the Landsat missions. Using Breaks for Additive Seasonal and Trend (BFAST), a generic change detection algorithm, we detect abrupt shifts in magnitude (immediate gain or loss in photosynthetic activity) and trend (greening or browning) of photosynthetic activity in the landscape surrounding each BBS route. Non-linear spatio-temporal models were used to correlate the proportion of changing land in the wider landscape with changes in local bird diversity.
