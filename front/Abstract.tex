% % Abstract

\thispagestyle{empty}
%\pdfbookmark[0]{Abstract}{Abstract} % Bookmark name visible in a PDF viewer
\addcontentsline{toc}{chapter}{\numberline{}Abstract}%

\begin{center}
%	\bigskip

    {\normalsize \myUni \\} % University name in capitals
    {\normalsize \myFaculty \\} % Faculty name
    {\normalsize \myDepartment \\} % Department name
    \bigskip\vspace*{.02\textheight}
    {\Large \textsc{Doctoral Thesis}}\par
    \bigskip
    
    {\rule{\linewidth}{1pt}\\%[0.4cm]
    \Large \myTitle \par} % Thesis title
    \rule{\linewidth}{1pt}\\[0.4cm]
    
    \bigskip
	{\Large by \myName \par} % Author name
    \bigskip\vspace*{.06\textheight}
\end{center}

    {\centering\Huge\textsc{\textbf{Abstract}} \par}
    \bigskip


    \noindent  \small Land is constantly changing because of natural and anthropogenic factors. One of the grand challenges facing humanity is the loss of biodiversity, caused by land change, which may affect ecosystem functioning. Attributes of land change, e.g. magnitude, time span, sequence or frequency, can be quantified reliably from remotely-sensed satellite data. Up to now, it was not clear how attributes of past land changes, e.g. those preceding biodiversity sampling, continue to influence local biodiversity across geographic regions and taxonomic groups. This thesis investigates the varying impacts of multiple attributes of land change on biodiversity globally by analysing links between broad-scale data on local biodiversity measures – calculated from the global PREDICTS database - and time series of different remotely-sensed satellite data from the period of 1984 to 2015. Overall past land changes were found to impact local  biodiversity more than present differences on land, however with considerable variability among taxonomic groups. Abrupt land changes of greater magnitude, that occurred more recently, reduced local biodiversity measures more, although biodiversity recovered as time passed. Furthermore, impacts of past land change varied depending on trajectories of land-cover types, affecting national and global biodiversity projections. While biodiversity change, quantified from time series of North American breeding bird surveys was correlated with, but not explained by, landscape-wide land changes, the frequency and magnitude of past, instead of concomitant, land changes was more important in explaining biodiversity change. These results indicate that global indicators of the impacts of land change on local biodiversity need to consider lasting influences of the past as ignoring them would result in incomplete assessments of biodiversity change. Remote sensing can assist in quantifying biologically-relevant attributes of land change in space and time, and such attributes should be incorporated into global assessments and projections of biodiversity change.
